\documentclass[]{article}
\usepackage{lmodern}
\usepackage{amssymb,amsmath}
\usepackage{ifxetex,ifluatex}
\usepackage{fixltx2e} % provides \textsubscript
\ifnum 0\ifxetex 1\fi\ifluatex 1\fi=0 % if pdftex
  \usepackage[T1]{fontenc}
  \usepackage[utf8]{inputenc}
\else % if luatex or xelatex
  \ifxetex
    \usepackage{mathspec}
    \usepackage{xltxtra,xunicode}
  \else
    \usepackage{fontspec}
  \fi
  \defaultfontfeatures{Mapping=tex-text,Scale=MatchLowercase}
  \newcommand{\euro}{€}
    \setmainfont{Calibri}
    \setsansfont{Gill Sans Light}
    \setmonofont[Mapping=tex-ansi]{Gill Sans Light}
\fi
% use upquote if available, for straight quotes in verbatim environments
\IfFileExists{upquote.sty}{\usepackage{upquote}}{}
% use microtype if available
\IfFileExists{microtype.sty}{%
\usepackage{microtype}
\UseMicrotypeSet[protrusion]{basicmath} % disable protrusion for tt fonts
}{}
\usepackage[a4paper]{geometry}
\usepackage{color}
\usepackage{fancyvrb}
\newcommand{\VerbBar}{|}
\newcommand{\VERB}{\Verb[commandchars=\\\{\}]}
\DefineVerbatimEnvironment{Highlighting}{Verbatim}{commandchars=\\\{\}}
% Add ',fontsize=\small' for more characters per line
\usepackage{framed}
\definecolor{shadecolor}{RGB}{48,48,48}
\newenvironment{Shaded}{\begin{snugshade}}{\end{snugshade}}
\newcommand{\KeywordTok}[1]{\textcolor[rgb]{0.94,0.87,0.69}{{#1}}}
\newcommand{\DataTypeTok}[1]{\textcolor[rgb]{0.87,0.87,0.75}{{#1}}}
\newcommand{\DecValTok}[1]{\textcolor[rgb]{0.86,0.86,0.80}{{#1}}}
\newcommand{\BaseNTok}[1]{\textcolor[rgb]{0.86,0.64,0.64}{{#1}}}
\newcommand{\FloatTok}[1]{\textcolor[rgb]{0.75,0.75,0.82}{{#1}}}
\newcommand{\CharTok}[1]{\textcolor[rgb]{0.86,0.64,0.64}{{#1}}}
\newcommand{\StringTok}[1]{\textcolor[rgb]{0.80,0.58,0.58}{{#1}}}
\newcommand{\CommentTok}[1]{\textcolor[rgb]{0.50,0.62,0.50}{{#1}}}
\newcommand{\OtherTok}[1]{\textcolor[rgb]{0.94,0.94,0.56}{{#1}}}
\newcommand{\AlertTok}[1]{\textcolor[rgb]{1.00,0.81,0.69}{{#1}}}
\newcommand{\FunctionTok}[1]{\textcolor[rgb]{0.94,0.94,0.56}{{#1}}}
\newcommand{\RegionMarkerTok}[1]{\textcolor[rgb]{0.80,0.80,0.80}{{#1}}}
\newcommand{\ErrorTok}[1]{\textcolor[rgb]{0.76,0.75,0.62}{{#1}}}
\newcommand{\NormalTok}[1]{\textcolor[rgb]{0.80,0.80,0.80}{{#1}}}
\usepackage{graphicx}
\makeatletter
\def\maxwidth{\ifdim\Gin@nat@width>\linewidth\linewidth\else\Gin@nat@width\fi}
\def\maxheight{\ifdim\Gin@nat@height>\textheight\textheight\else\Gin@nat@height\fi}
\makeatother
% Scale images if necessary, so that they will not overflow the page
% margins by default, and it is still possible to overwrite the defaults
% using explicit options in \includegraphics[width, height, ...]{}
\setkeys{Gin}{width=\maxwidth,height=\maxheight,keepaspectratio}
\ifxetex
  \usepackage[setpagesize=false, % page size defined by xetex
              unicode=false, % unicode breaks when used with xetex
              xetex]{hyperref}
\else
  \usepackage[unicode=true]{hyperref}
\fi
\hypersetup{breaklinks=true,
            bookmarks=true,
            pdfauthor={ManyLabs2 (Corresponding coder: Fred Hasselman)},
            pdftitle={ManLabs2 R Code Review},
            colorlinks=true,
            citecolor=blue,
            urlcolor=blue,
            linkcolor=magenta,
            pdfborder={0 0 0}}
\urlstyle{same}  % don't use monospace font for urls
\setlength{\parindent}{0pt}
\setlength{\parskip}{6pt plus 2pt minus 1pt}
\setlength{\emergencystretch}{3em}  % prevent overfull lines
\setcounter{secnumdepth}{0}

%%% Use protect on footnotes to avoid problems with footnotes in titles
\let\rmarkdownfootnote\footnote%
\def\footnote{\protect\rmarkdownfootnote}

%%% Change title format to be more compact
\usepackage{titling}

% Create subtitle command for use in maketitle
\newcommand{\subtitle}[1]{
  \posttitle{
    \begin{center}\large#1\end{center}
    }
}

\setlength{\droptitle}{-2em}
  \title{ManLabs2 R Code Review}
  \pretitle{\vspace{\droptitle}\centering\huge}
  \posttitle{\par}
  \author{\href{https://osf.io/8cd4r}{ManyLabs2} (Corresponding coder:
\href{https://osf.io/ujgs6/}{Fred Hasselman})}
  \preauthor{\centering\large\emph}
  \postauthor{\par}
  \predate{\centering\large\emph}
  \postdate{\par}
  \date{3 February 2016}



\begin{document}

\maketitle


{
\hypersetup{linkcolor=black}
\setcounter{tocdepth}{4}
\tableofcontents
}
\section{Instruction for reviewers}\label{instruction-for-reviewers}

\subsection{Analysis Strategy}\label{analysis-strategy}

The ManyLabs2 data analysis strategy attempts to regard three principles
that maximize research transparency:

\begin{enumerate}
\def\labelenumi{\arabic{enumi}.}
\item
  \textbf{Principle of Equality}: All data should be treated equally by
  a code. That is, the code should do its job generating results while
  at the same time being as naive as possible to the particular facts of
  the study being analysed. This will reduce any chances of bias with
  respect to the outcomes of a certain dataset or a particular study. If
  it is necessary to add study specific code, the second principle
  should be regarded.
\item
  \textbf{Principle of Transparency}: All operations that are crucial
  for obtaining an analysis result should be available for inspection by
  anyone who wishes to do so. This should be possible without the help
  of the auhtors that generated the code. The operations concern the
  application of data filtering rules, computation of variables derived
  from original measurements, running an analysis and constructing
  graphs, tables and figures. If full transparency is not possible, the
  third principle should be regarded.
\item
  \textbf{Principle of Reproducibility}: The most basic requirement for
  analysis results is that they should be reproducable given the
  original code and the original data set. However, any new
  implementation of the same analysis strategy in a different context,
  or application of the code to a different dataset, e.g.~a replication
  study, should not be problematic. That is, outcomes may differ between
  data sets, but this should not be attributable to any details of the
  code or the analysis strategy.
\end{enumerate}

\subsection{\texttt{R} as a parser of online
code.}\label{r-as-a-parser-of-online-code.}

The
\href{https://docs.google.com/document/d/1B5sVz3jlKMGnpxik-cfCeR8ngeQ249bzQExaVd5FzZM/edit}{pre-registered
Manylabs2 protocol} describes a number of analyses per replication study
that can be categorised as \textbf{Primary} (target replications per
site), \textbf{Secondary} (additional analyses per site, e.g.~on
subgroups), and \textbf{Global} (analyses on the entire dataset).

These \emph{promised analyses} have all been implemented in \texttt{R}
in a transparent way and this implementation is now ready for an
independent review.

\subsection{Implementation}\label{implementation}

Functions avalaible in an
\href{https://github.com/FredHasselman/ManyLabRs/tree/master/manylabRs}{\texttt{R}
package on GitHub} extract information and instructions about each
promised analysis a table that is openly accessible, the
\href{https://docs.google.com/spreadsheets/d/1fqK3WHwFPMIjNVVvmxpMEjzUETftq_DmP5LzEhXxUHA/edit\#gid=769239110}{masteRkey}
spreadsheet.

Each row in the table represents an analysis, the columns contain
specific information about the analysis:

\begin{itemize}
\item
  Columns A through E are identifiers for study, analysis and slate.
\item
  Column F and G contain \texttt{R} commands which will extract and
  label the columnms from the dataset needed for the analysis.
\item
  Column H and I contain filter instructions for cases and subsamnples.
\item
  Columns J through L contain information about the nature of the
  analysis (Global, Primary Secondary).
\item
  Column M lists the name of a analysis specific \emph{variable
  function} (\texttt{varfun.}) which in most cases just reorganises the
  variables specified in previous columns so they can be passed to the
  analysis code. In some cases these function perform specific
  calculations required by the original analyses.
\item
  Columns N through S contain information about the statistiscal tests
\end{itemize}

\subsection{What is in need of review by \texttt{R}
experts?}\label{what-is-in-need-of-review-by-r-experts}

The code runs and perfoms analyses on the data, that much we know
\texttt{:)}.

We would like to get your expert opinion on the following:

\begin{enumerate}
\def\labelenumi{\arabic{enumi}.}
\itemsep1pt\parskip0pt\parsep0pt
\item
  Does the \texttt{R} code in column O (\texttt{stat.test}) reflect the
  promised analyses in the protocol?

  \begin{itemize}
  \itemsep1pt\parskip0pt\parsep0pt
  \item
    In order to evaluate this you'll need to look at the analysis plan
    for a specific study in the protocol and figure out whether the
    \texttt{R} code in rows of column O for that study represents all
    the tests that is described there.\\
  \item
    In many cases this will be straightforward, without any need to
    actually run any code ooking at the way in which the variables are
    grouped and labelled and, filters are applied and which settings are
    used for the analysis, e.g.~direction of the test (column P,
    \texttt{stat.params}).
  \item
    In some cases you will need to inspect the contents of the
    \texttt{varfun} listed in column M. All varfuns are available in a
    sourceable file on Github. There is also a pdf of the
    \href{https://github.com/FredHasselman/ManyLabRs/blob/master/ManyLabRs.pdf}{manual
    pages}
  \end{itemize}
\end{enumerate}

\begin{Shaded}
\begin{Highlighting}[]
\KeywordTok{require}\NormalTok{(devtools)}
\KeywordTok{source_url}\NormalTok{(}\StringTok{"https://raw.githubusercontent.com/FredHasselman/ManyLabRs/master/manylabRs/R/ML2_variable_functions.R"}\NormalTok{)}
\end{Highlighting}
\end{Shaded}

\begin{enumerate}
\def\labelenumi{\arabic{enumi}.}
\setcounter{enumi}{1}
\itemsep1pt\parskip0pt\parsep0pt
\item
  Does the output correspond to what may be expected by the \texttt{R}
  code in column O?

  \begin{itemize}
  \itemsep1pt\parskip0pt\parsep0pt
  \item
    Think about the test-statistic, the parameters (df) and N.
  \item
    The results files can be found in the dropbox associated with a
    \href{https://osf.io/fprzu/}{private project on OSF}:
    \texttt{/TestOutput/}

    \begin{itemize}
    \itemsep1pt\parskip0pt\parsep0pt
    \item
      \texttt{RAW.DATA.PRIVATE} - Contains the output of the data
      merging and filtering rules described in
      \href{http://fredhasselman.com/htmlHost/ManyLabs/ML2_data_cleaning.html}{\texttt{ML2\_data\_cleaning.html}}
    \end{itemize}
  \end{itemize}
\end{enumerate}

\begin{center}\rule{0.5\linewidth}{\linethickness}\end{center}

\section{Appendix}\label{appendix}

The analysis \texttt{cor.test.fisherZ()} defined as follows:

\begin{Shaded}
\begin{Highlighting}[]
\NormalTok{cor.test.fisherZ <-}\StringTok{ }\NormalTok{function(}\DataTypeTok{r1=}\OtherTok{NULL}\NormalTok{,}\DataTypeTok{r2=}\OtherTok{NULL}\NormalTok{,}\DataTypeTok{n1=}\OtherTok{NULL}\NormalTok{,}\DataTypeTok{n2=}\OtherTok{NULL}\NormalTok{,}\DataTypeTok{p=}\OtherTok{TRUE}\NormalTok{)\{}
    \NormalTok{if((}\KeywordTok{dim}\NormalTok{(}\KeywordTok{as.matrix}\NormalTok{(r1))[}\DecValTok{2}\NormalTok{]==}\DecValTok{2}\NormalTok{)&(}\KeywordTok{dim}\NormalTok{(}\KeywordTok{as.matrix}\NormalTok{(r2))[}\DecValTok{2}\NormalTok{]==}\DecValTok{2}\NormalTok{))\{}
        \NormalTok{r1 <-}\StringTok{ }\KeywordTok{cor}\NormalTok{(r1[,}\DecValTok{1}\NormalTok{],r1[,}\DecValTok{2}\NormalTok{],}\DataTypeTok{use=}\StringTok{"pairwise.complete.obs"}\NormalTok{)}
        \NormalTok{r2 <-}\StringTok{ }\KeywordTok{cor}\NormalTok{(r2[,}\DecValTok{1}\NormalTok{],r2[,}\DecValTok{2}\NormalTok{],}\DataTypeTok{use=}\StringTok{"pairwise.complete.obs"}\NormalTok{)}
    \NormalTok{\} else\{}
        \NormalTok{if(}\KeywordTok{all}\NormalTok{((}\KeywordTok{dim}\NormalTok{(}\KeywordTok{as.matrix}\NormalTok{(r1))!=}\DecValTok{1}\NormalTok{))&}\KeywordTok{all}\NormalTok{(}\KeywordTok{dim}\NormalTok{(}\KeywordTok{as.matrix}\NormalTok{(r2))!=}\DecValTok{1}\NormalTok{))\{}
            \KeywordTok{disp}\NormalTok{(}\DataTypeTok{message =} \StringTok{"r1 and r2 each need to be:"}\NormalTok{, }\DataTypeTok{header =} \StringTok{"cor.test.fisherZ"}\NormalTok{, }\DataTypeTok{footer =} \OtherTok{FALSE}\NormalTok{)}
            \KeywordTok{disp}\NormalTok{(}\DataTypeTok{message =} \StringTok{"- Either a single numerical value representing a correlation,"}\NormalTok{, }\DataTypeTok{header =} \OtherTok{FALSE}\NormalTok{, }\DataTypeTok{footer =} \OtherTok{FALSE}\NormalTok{)}
            \KeywordTok{disp}\NormalTok{(}\DataTypeTok{message =} \StringTok{"- Or a 2 column matrix from which a correlation r1 and r2 can be calculated"}\NormalTok{, }\DataTypeTok{header =} \OtherTok{FALSE}\NormalTok{)}
        \NormalTok{\}}
    \NormalTok{\}}
    \NormalTok{z <-}\StringTok{ }\NormalTok{((}\KeywordTok{atanh}\NormalTok{(r1)-}\KeywordTok{atanh}\NormalTok{(r2))/((}\DecValTok{1}\NormalTok{/(n1}\DecValTok{-3}\NormalTok{))+(}\DecValTok{1}\NormalTok{/(n2}\DecValTok{-3}\NormalTok{)))^}\FloatTok{0.5}\NormalTok{)}
    \NormalTok{if(p)\{p<-}\DecValTok{2}\NormalTok{*(}\DecValTok{1}\NormalTok{-}\KeywordTok{pnorm}\NormalTok{(}\KeywordTok{abs}\NormalTok{(z)))\} else \{p=}\OtherTok{NULL}\NormalTok{\}}
    \NormalTok{stat.test <-}\StringTok{ }\KeywordTok{structure}\NormalTok{(}\KeywordTok{list}\NormalTok{(}\DataTypeTok{statistic =} \NormalTok{z,}
                                \DataTypeTok{method=}\StringTok{"Fisher r-to-Z transformed test for 2 independent correlations"}\NormalTok{,}
                                \DataTypeTok{parameter=}\NormalTok{n1+n2,}
                                \DataTypeTok{p=}\NormalTok{p)}
    \NormalTok{)}
    \KeywordTok{class}\NormalTok{(stat.test) <-}\StringTok{ "htest"}
    \KeywordTok{return}\NormalTok{(stat.test)}
\NormalTok{\}}
\end{Highlighting}
\end{Shaded}

The function \texttt{any2r()} is defined as follows:

\begin{Shaded}
\begin{Highlighting}[]
\CommentTok{#' any2r}
\CommentTok{#'}
\CommentTok{#' Converts most common test statistics into (signed) effect size \textbackslash{}code\{r\}.}
\CommentTok{#'}
\CommentTok{#' @param s     Value(s) of a test statistic.}
\CommentTok{#' @param df1     Degrees of freedom}
\CommentTok{#' @param df2     NULL or degrees of freedom of the denominator for the f-distribution.}
\CommentTok{#' @param N     Number of data points used in calculation of test-statistic.}
\CommentTok{#' @param esType     Type of test statistic. One of: "t", "lm.t", "f", "lm.f", "r", "X2", "Z", "lm.Z"}
\CommentTok{#' @param CIcalc     If \textbackslash{}code\{TRUE\} (default) the Confidence Interval for the test statistic in \textbackslash{}code\{x\} will be calculated using the "Confidence limits for noncentral parameters" functions in package (e.g., for type - "t": \textbackslash{}link[MBESS]\{conf.limits.nct\}).}
\CommentTok{#' @param CL    Confidence Limit (default: .95).}
\CommentTok{#' @param keepSign     Return effect size with sign of test statistic? (default = TRUE).}
\CommentTok{#'}
\CommentTok{#' @details The prefix "lm" is currently disregarded, but will be implemented in future versions to indicate the test statistic is in fact a fixed factor in a linear model.}
\CommentTok{#'}
\CommentTok{#' @author}
\CommentTok{#' CHJ Hartgerink (original code)}
\CommentTok{#' F Hasselman (added vector input and CI calculation )}
\CommentTok{#'}
\CommentTok{#' @return The effect size \textbackslash{}code\{r\} corresponding to the test statistic(s).}
\CommentTok{#' @export}
\CommentTok{#'}
\CommentTok{#' @examples}
\NormalTok{any2r <-}\StringTok{ }\NormalTok{function(st, }\DataTypeTok{df1 =} \OtherTok{NULL}\NormalTok{, }\DataTypeTok{df2 =} \OtherTok{NULL}\NormalTok{, }\DataTypeTok{N =} \OtherTok{NULL}\NormalTok{, }\DataTypeTok{esType =} \OtherTok{NA}\NormalTok{, }\DataTypeTok{CIcalc =} \OtherTok{TRUE}\NormalTok{, }\DataTypeTok{CL =} \NormalTok{.}\DecValTok{95}\NormalTok{, }\DataTypeTok{keepSign =} \OtherTok{TRUE}\NormalTok{)\{}
    \KeywordTok{require}\NormalTok{(MBESS)}
    \NormalTok{esType <-}\StringTok{ }\KeywordTok{gsub}\NormalTok{(}\StringTok{"lm."}\NormalTok{,}\StringTok{""}\NormalTok{,esType)}
    \NormalTok{if(CIcalc)\{}
        \NormalTok{sCI <-}\StringTok{ }\KeywordTok{get.ncpCI}\NormalTok{(st, df1, df2, N, esType, CL)}
    \NormalTok{\} else \{}
        \NormalTok{sCI <-}\StringTok{ }\NormalTok{st}
    \NormalTok{\}}

    \NormalTok{esComp <-}\StringTok{ }\KeywordTok{list}\NormalTok{()}
    \NormalTok{cnt <-}\StringTok{ }\DecValTok{0}

    \NormalTok{for(cnt in }\KeywordTok{seq_along}\NormalTok{(sCI))\{}

        \NormalTok{x <-}\StringTok{ }\NormalTok{sCI[cnt]}

        \NormalTok{esComp[[cnt]] <-}\StringTok{ }\KeywordTok{ifelse}\NormalTok{(}\KeywordTok{grepl}\NormalTok{(}\StringTok{"t"}\NormalTok{,esType), }\KeywordTok{sqrt}\NormalTok{((x^}\DecValTok{2}\NormalTok{*(}\DecValTok{1} \NormalTok{/}\StringTok{ }\NormalTok{df1)) /}\StringTok{ }\NormalTok{(((x^}\DecValTok{2}\NormalTok{*}\DecValTok{1}\NormalTok{) /}\StringTok{ }\NormalTok{df1) +}\StringTok{ }\DecValTok{1}\NormalTok{)),}
                                \KeywordTok{ifelse}\NormalTok{(}\KeywordTok{grepl}\NormalTok{(}\StringTok{"f"}\NormalTok{,esType), }\KeywordTok{sqrt}\NormalTok{((x*(df2 /}\StringTok{ }\NormalTok{df1)) /}\StringTok{ }\NormalTok{(((x*df2) /}\StringTok{ }\NormalTok{df1) +}\StringTok{ }\DecValTok{1}\NormalTok{))*}\KeywordTok{sqrt}\NormalTok{(}\DecValTok{1}\NormalTok{/df2),}
                                       \KeywordTok{ifelse}\NormalTok{(}\KeywordTok{grepl}\NormalTok{(}\StringTok{"r"}\NormalTok{,esType), x,}
                                              \KeywordTok{ifelse}\NormalTok{(}\KeywordTok{grepl}\NormalTok{(}\StringTok{"X2"}\NormalTok{,esType), }\KeywordTok{sqrt}\NormalTok{(x/N),}
                                                     \KeywordTok{ifelse}\NormalTok{(}\KeywordTok{grepl}\NormalTok{(}\StringTok{"Z"}\NormalTok{,esType), }\KeywordTok{tanh}\NormalTok{(x *}\StringTok{ }\KeywordTok{sqrt}\NormalTok{(}\DecValTok{1}\NormalTok{/(N}\DecValTok{-3}\NormalTok{))),}
                                                            \OtherTok{NA}\NormalTok{)}
                                              \NormalTok{)}
                                       \NormalTok{)}
                                \NormalTok{)}
        \NormalTok{)}
    \NormalTok{\}}

    \NormalTok{ES <-}\StringTok{ }\KeywordTok{cbind}\NormalTok{(sCI,}\KeywordTok{rbind}\NormalTok{(}\KeywordTok{unlist}\NormalTok{(esComp)))}
    \KeywordTok{colnames}\NormalTok{(ES) <-}\StringTok{ }\KeywordTok{c}\NormalTok{(}\KeywordTok{c}\NormalTok{(}\StringTok{"ncp"}\NormalTok{,}\StringTok{"ncp.lo"}\NormalTok{,}\StringTok{"ncp.hi"}\NormalTok{)[}\DecValTok{1}\NormalTok{:cnt],}\KeywordTok{c}\NormalTok{(}\StringTok{"r"}\NormalTok{,}\StringTok{"r.ciL"}\NormalTok{,}\StringTok{"r.ciU"}\NormalTok{)[}\DecValTok{1}\NormalTok{:cnt])}
    \NormalTok{if(keepSign)\{}
        \NormalTok{ES <-}\StringTok{ }\NormalTok{ES *}\StringTok{ }\KeywordTok{c}\NormalTok{(}\KeywordTok{c}\NormalTok{(}\DecValTok{1}\NormalTok{,}\DecValTok{1}\NormalTok{,}\DecValTok{1}\NormalTok{)[}\DecValTok{1}\NormalTok{:cnt], }\KeywordTok{sign}\NormalTok{(sCI)[}\DecValTok{1}\NormalTok{:cnt])}
    \NormalTok{\}}
    \KeywordTok{return}\NormalTok{(ES)}
\NormalTok{\}}


\NormalTok{get.ncpCI <-}\StringTok{ }\NormalTok{function(x, df1, df2, N, esType, }\DataTypeTok{CL=}\NormalTok{.}\DecValTok{95}\NormalTok{)\{}
    \KeywordTok{require}\NormalTok{(MBESS)}
    \NormalTok{esType <-}\StringTok{ }\KeywordTok{gsub}\NormalTok{(}\StringTok{"lm."}\NormalTok{,}\StringTok{""}\NormalTok{,esType)}
    \NormalTok{ncCI <-}\StringTok{ }\KeywordTok{list}\NormalTok{()}
    \NormalTok{switch(esType,}
           \DataTypeTok{t =} \NormalTok{ncCI <-}\StringTok{ }\NormalTok{MBESS::}\KeywordTok{conf.limits.nct}\NormalTok{(}\DataTypeTok{t.value=}\NormalTok{x, }\DataTypeTok{conf.level=}\NormalTok{CL, }\DataTypeTok{df=}\NormalTok{df1),}
           \DataTypeTok{f =} \NormalTok{ncCI <-}\StringTok{ }\NormalTok{MBESS::}\KeywordTok{conf.limits.ncf}\NormalTok{(}\DataTypeTok{F.value=}\NormalTok{x, }\DataTypeTok{conf.level=}\NormalTok{CL, }\DataTypeTok{df.1=}\NormalTok{df1, }\DataTypeTok{df.2=}\NormalTok{df2),}
           \DataTypeTok{r =} \NormalTok{ncCI <-}\StringTok{ }\KeywordTok{list}\NormalTok{(}\DataTypeTok{Lower.Limit =} \KeywordTok{ci.R}\NormalTok{(}\DataTypeTok{R=}\NormalTok{x, }\DataTypeTok{conf.level=}\NormalTok{CL, }\DataTypeTok{N=}\NormalTok{(df1}\DecValTok{+2}\NormalTok{), }\DataTypeTok{K=}\DecValTok{1}\NormalTok{)$Lower.Conf.Limit.R,}
                            \DataTypeTok{Upper.Limit =} \KeywordTok{ci.R}\NormalTok{(}\DataTypeTok{R=}\NormalTok{x, }\DataTypeTok{conf.level=}\NormalTok{CL, }\DataTypeTok{N=}\NormalTok{(df1}\DecValTok{+2}\NormalTok{), }\DataTypeTok{K=}\DecValTok{1}\NormalTok{)$Upper.Conf.Limit.R),}
           \DataTypeTok{X2 =} \NormalTok{ncCI <-}\StringTok{ }\NormalTok{MBESS::}\KeywordTok{conf.limits.nc.chisq}\NormalTok{(}\DataTypeTok{Chi.Square=}\NormalTok{x, }\DataTypeTok{df=}\NormalTok{df1,}
                                                    \DataTypeTok{alpha.lower=}\DecValTok{0}\NormalTok{, }\DataTypeTok{alpha.upper=}\DecValTok{1}\NormalTok{-CL, }\DataTypeTok{conf.level=}\OtherTok{NULL}\NormalTok{),}
           \DataTypeTok{Z =} \NormalTok{ncCI <-}\StringTok{ }\KeywordTok{list}\NormalTok{(}\DataTypeTok{Lower.Limit =} \NormalTok{(x -}\StringTok{ }\KeywordTok{qnorm}\NormalTok{((}\DecValTok{1}\NormalTok{-CL)/}\DecValTok{2}\NormalTok{, }\DataTypeTok{lower.tail =} \OtherTok{FALSE}\NormalTok{)),}
                            \DataTypeTok{Upper.Limit =} \NormalTok{(x +}\StringTok{ }\KeywordTok{qnorm}\NormalTok{((}\DecValTok{1}\NormalTok{-CL)/}\DecValTok{2}\NormalTok{, }\DataTypeTok{lower.tail =} \OtherTok{FALSE}\NormalTok{)))}
    \NormalTok{)}
    \KeywordTok{return}\NormalTok{(}\KeywordTok{cbind}\NormalTok{(}\DataTypeTok{ncp =} \NormalTok{x,}
                 \DataTypeTok{ncp.lo =} \NormalTok{ncCI$Lower.Limit,}
                 \DataTypeTok{ncp.hi =} \NormalTok{ncCI$Upper.Limit)}
    \NormalTok{)}
\NormalTok{\}}
\end{Highlighting}
\end{Shaded}

\end{document}
